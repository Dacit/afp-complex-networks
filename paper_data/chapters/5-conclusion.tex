\section{Discussion}\label{sec:conclusion}
We analyzed the formal entity networks induced by Isabelle/AFP entries
and found that they exhibit small-world effects (i.e., have short average path lengths and high clustering coefficients)
and are weakly scale-free.
Hence, we attempted to connect those networks with research from complex networks of software systems,
to obtain quality indicators and to extract important concepts.
While network centrality metrics and fractal dimension (on a global level) work well as software defect indicators,
we found that this is not the case for formalization networks,
and the much simpler SLOC metric is superior to all the above.
However, we note that in software systems, the absolute number of defects is used as ground-truth for quality,
but frequently used components have a higher chance to incur a fault even if the underlying code quality is the same.
Hence, defect count is dependent on centrality, whereas lint frequency is independent of such an effect.
In contrast, extracting important concepts works quite well.
We achieved best results (in terms of \fOne-score) by selecting the top-\num{\bestPredOptN} entities with respect to in-degree,
where \SI[round-mode=places,round-precision=1]{\bestPredOptRec}{\percent} of defining positions
identified as most important by article authors could be identified with a precision of \SI[round-mode=places,round-precision=1]{\bestPredOptPre}{\percent}.
A second sweet spot is attained for the top-\num{\bestPredOptTwoN} entities, where \SI[round-mode=places,round-precision=1]{\bestPredOptTwoRec}{\percent} of the relevant defining positions can be retrieved
with \SI[round-mode=places,round-precision=1]{\bestPredOptTwoPre}{\percent} precision.

\subsection{Threats to Validity}
The combinatorial explosion of parameter and experiment spaces poses the biggest threat to validity of this research.
There are many ways of constructing formal entity graphs
(for example, restricting the graph to just theorems and proof dependencies),
multiple levels of abstractions (entities, defining positions, theories, sessions, entries),
and many sub-graphs that can be considered (per-entry, full, full without base logic).
Moreover, a large number of other graph metrics from social network analysis exist that one could utilize,
though it is found that many of those are quite similar and strongly correlate with each other~\cite{MetricsOO2010Concas}.
Furthermore, the large scale and variety of the involved formal entity networks make experiments quite time-consuming.
We cover a reasonable portion of that parameter space and orientate ourselves at related research,
but there might still be unexplored territory where better results are attainable.

\subsection{Future work}
In this research,
we only considered simple metrics and statistical correlations on Isabelle dependency graphs.
With this basis, the rich body of data would also allow building and evaluating more intricate models,
for instance deep learning models such as graph neural networks,
towards which much of the research nowadays has gravitated.
Recently, they were found to work quite well for defect prediction in software systems~\cite{GNNSoftwareDefects2022Sikic}.

Additionally,
the network analysis itself is not Isabelle-specific,
and could be transferred to other systems as well.
While we have not attempted that for the work presented here,
one would expect the results to carry over at least to other classical prover systems.
Verifying this assumption would be a worthwhile contribution---%
building the integration with other systems is the main challenge.

Next, while we could successfully extract important concepts from the dependency graph,
more engineering work is required to put them in practice:
Overviews for AFP entries need to be generated and added to the AFP website,
which requires proper integration into the existing tooling and pipelines.
Moreover, generating this overview dynamically in a running prover session
would be a valuable add-on tool for Isabelle prover IDEs.

Furthermore, there are other applications of network structure that have not yet been examined.
For instance, recommendations for lemma placement could be generated by utilizing community detection and finding entities that are not in the theory in which their community is located.
Another application is in the automated theorem prover integration,
where facts from the background theory that might be useful in solving a given goal need to be selected and ranked.
This could be improved by utilizing graph knowledge.

Lastly, visualizations of the (aggregated) network graphs could be a valuable tool in making the vast formalization landscapes more navigable.