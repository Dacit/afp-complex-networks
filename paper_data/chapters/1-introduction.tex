\section{Introduction}
Interactive proof assistants (ITPs) have become quite mature,
to a point were even novel research can be performed in a theorem prover,
such as the recently completed proof of the Clausen-Scholze theorem in Lean~\cite{Scholze2021Scholze}.
At the same time,
not much is known empirically about the structure of formalizations.

Due to long-term community efforts,
large-scale collections of formalized material exist in some (earlier) systems
such as \emph{Mizar}~\cite{Mizar1992Rudnicki} (\emph{Mizar Mathematical Library}) and \emph{Isabelle}~\cite{Isabelle1998Paulson} (\emph{Archive of Formal Proofs}).
At the time of writing, the Archive of Formal Proofs (AFP) consists of over three million lines of code,
and more than \num{195000} lemmas have been proven
in its \num{675} different articles (called \emph{entries})~\cite{Statistics2022Afp}.
Entries are characterized by their metadata (abstract, authors, topics, etc.)
and Isabelle theories, with are structured into \emph{sessions} (usually a single one per entry).

At the mark of one million lines of code,
the size and authorship distribution in the AFP, and how those evolved over time, has been analyzed~\cite{MiningAFP2015Blanchette},
as well as some (mostly) syntactic properties.
It was found that proofs made up \SI{58}{\percent} of the code,
and the total proof size has a roughly quadratic relationship with the number of constants in the goal.
However, except for some insights into the entry import-graph
(where it became apparent that only very few entries were re-used),
the syntactical analysis tells us very little about the underlying formalization structure.

\textbf{Problem}.
As libraries and archives expand and ITP use becomes more widespread,
understanding the formalization structure becomes critical.
Due to the vastness of formalization archives,
it is important to aid users in grasping material faster and more easily.
Moreover, building high-quality formalizations
that are re-usable and extensible in the first place is key.

\textbf{Solution}.
The related problems in the field of software engineering are widely studied.
It is found that methodology from \emph{complex network science} ---
which is concerned with the structural properties of networks from all kinds of empirical research ---
is applicable to a wide range of problems
and often superior to analysis of software-specific characteristics~\cite{DefectsMetrics2008Zimmermann,FractalDimension2013Turnu}.
This gives rise to analyzing the dependency graphs of formalizations
to determine whether complex network methodology is applicable
and which concrete questions can be answered.

\textbf{Contribution}.
In this work, we investigate the formalization network of the Isabelle AFP
for patterns commonly found in complex networks.
We also examine two concrete problems, namely,
assessing formalization quality,
and identifying the most important parts of formalizations.
Our source code and data is publicly available\footnote{\url{https://github.com/dacit/afp-complex-networks}}.

\textbf{Organization}.
In Section~\ref{sec:background},
we give an overview about concepts of complex networks,
and discuss in Section~\ref{sec:related_software} how they are used in related work in software engineering.
We report our analysis and findings in Section~\ref{sec:network},
and discuss our conclusions in Section~\ref{sec:conclusion}.