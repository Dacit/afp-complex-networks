\section{Complex Network Concepts}\label{sec:background}
Complex systems can frequently be modeled as graphs.
Of those, many real-world networks from biology, social science, engineering, and information systems,
are found to exhibit certain structural properties that do not appear in graphs generated from simple models,
and thus called \emph{complex networks}.
In general, those graphs are directed,
though some complex network research only considers undirected graphs.
We will consider the general case unless stated otherwise.

In the following, we explain three of the most important complex network properties,
as well as graph metrics found useful in complex networks.
We express graphs by their adjacency matrix $A$, i.e., $a_{ij}$ denotes an edge between $i$ and $j$ (we do not consider multiplicities).
Thus, in-degree $k_\text{in}$ of a node $i$ can be defined as follows
(out-degree $k_\text{out}$ and overall degree $k$ similarly):
\begin{equation}
    k_\text{in}(i) = \sum_{j\neq i}{a_{ji}}
\end{equation}

The \emph{small-world} property~\cite{SmallWorld1998Watts} describes networks which are in between regular networks
(where every node has the same degree, e.g., lattice graphs),
and random graphs where edges are created uniformly at random (Erd\H{o}s--R\'enyi (ER) model~\cite{RandomGraph1959Erdos}).
Small-world networks exhibit average path lengths
that are short like in ER-graphs---%
in contrast to regular graphs, where average path length increases linearly with size---%
but comparatively strong clustering, which is low in ER-graphs but high in many regular networks.
A measure for clustering, the \emph{clustering coefficient} $CC$,
is defined as the likelihood that a node's neighbors are also connected to each other in the undirected graph,
i.e.:
\begin{equation}
    CC(i) = \frac{\sum_{i\neq j,j\neq l,l\neq i}{a_{ij}a_{jl}a_{li}}}{2\cdot k(i)(k(i) - 1)}
\end{equation}

Secondly, the \emph{scale-free} property~\cite{ScaleFree1999Albert} describes networks that have no characteristic scale,
i.e., they have similarly structure no matter the size.
In particular,
the \emph{in-degree} distribution of scale-free networks follows a power-law,
for sufficiently large $k_\text{in}$ (and up to a maximal degree for finite networks):
\begin{equation}
    \Pr[k_\text{in}=x] \propto x^{-\alpha}
\end{equation}
In empirical scale-free networks, $\alpha$ (which represents the slope on the linear $\log$--$\log$ plot) is typically in the range of \numrange{2}{3}~\cite{PowlawEmpiric2009Clauset}.

The third important concept is the notion of self-similarity,
which underlies many generative processes found in nature.
In complex networks,
self-similarity can be measured by the \emph{fractal dimension} $d_B$~\cite{FractalComplex2005Song}, which is the exponent of the power-law relationship:
\begin{equation}
    N_B(l_B) \propto l_B^{-d_B}
\end{equation}
where $N_B$ is the minimal number of connected components that the network can be split into
such that each node is reachable from any other within $l_B$ steps (in the undirected graph).

In addition to in- and out-degree,
we use the following network centrality measures
(cf.~\cite{SnaMetrics2003Newman}) in this work:
\emph{Eigenvector centrality} $C_\lambda$ is the (normalized) solution to
$\bm{A}\bm{x} = \lambda\bm{x}$
with non-negative components where $\lambda$ is the largest eigenvalue of the adjacency matrix
(which uniquely exists since all entries are non-negative).
Similar to PageRank~\cite{pagerank}, it yields the importance of a node in the network.
\emph{Closeness centrality} is the average distance from a node to every other node.
However, since it is not defined for unconnected graphs, we use harmonic closeness centrality $C_\text{dist}$.
With $d_{ij}$ denoting the shortest distance between $i$ and $j$ (and $n$ the total number of nodes):
\begin{equation}
    C_\text{dist}(i) = \frac{1}{n-1}\sum_{j\neq i,d_{ij}<\infty}{\frac{1}{d_{ij}}}
\end{equation}
\emph{Betweenness centrality} $C_p$ measures how many shortest paths $p_{ij}$ (from $i$ to $j$) go through a node, i.e., it captures how much a node behaves as a hub in the network:
\begin{equation}% \sum  min_i. _p_ = [p1, ..., pn]
    C_p(i) = \frac{|\{p_{kl}|i\in p_{kl}\}|}{|\{p_{kl}\}|}
\end{equation}
