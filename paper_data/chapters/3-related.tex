\section{Complex Network Science in Software Engineering}\label{sec:related_software}

For software systems,
Valverde et al.~\cite{ScaleFreeSoftware2002Valverde} first discovered scale-free and small-world behavior and ascribed it to modularity requirements.
Myers~\cite{Networks2003Myers} found class graphs and procedure calls to be weakly scale-free,
and attributed those properties to object-oriented (OO) design.
However, rigorous statistical testing was not performed
(and in fact, for most networks claimed to be scale-free, other distributions are more likely~\cite{ScalefreeRare2019Broido}).

Fractal dimension has been studied for software systems by Concas et al.~\cite{FracDimSoftware2006Concas},
where $d_B$ was found to usually be in the range of \numrange{3}{5}.
Moreover, Turnu et al.~\cite{FractalDimension2013Turnu} found the fractal dimension to be strongly correlated with OO software quality indicators
and with the number of defects that later occurred,
whereas correlations with OO quality metrics would not be significant at the common significance level of $p = 0.05$.

Network centrality measures  have been found to be suitable for defect prediction as well:
Zimmermann and Nagappan~\cite{DefectsMetrics2008Zimmermann} found that out-degree and eigenvector centrality (among others) are more strongly correlated than OO metrics,
but slightly worse than other software engineering metrics;
overall, the correlation is of medium strength.
However, for predicting central binaries,
network centrality measures greatly outperformed other complexity measures.
These results could later only be confirmed for larger-scale projects (on a source-code level)~\cite{NetworkDefects2009Tosun}.

\v{S}ubelj and Bajec~\cite{CommunityDetect2011Subelj} have used complex network methodology for community detection,
i.e., to find structurally grouped modules.
Later, they successfully employed the detection on java class graphs to infer the package structure with high accuracy
as well as to separate the graph into structurally similar partitions,
which could be used to generate refactoring recommendations~\cite{Networks2012Subelj,Clustering2012Subelj}.