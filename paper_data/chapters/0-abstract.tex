\begin{abstract}
Formalization libraries for interactive theorem provers are rapidly growing in size,
but only little is understood structurally about those developments.
We aim to address the arising challenges by utilizing dependency graphs of the underlying formal entities.
For the Isabelle Archive of Formal Proofs, the individual entry graphs consist of \SI[round-mode=places,round-precision=1]{\numNodes}{\million} nodes and \SI[round-mode=places,round-precision=1]{\numEdges}{\million} edges in total,
and exhibit certain complex network characteristics:
Node in-degrees weakly follow scale-free distributions with an average exponent of $\alpha=\num[round-mode=places,round-precision=2]{\degreeIndAlpha}$,
and the high clustering coefficient (avg.\ \num[round-mode=places,round-precision=2]{\avgCC})
together with the short average path length (\num[round-mode=places,round-precision=2]{\avgL})
indicate small-world effects.
We did not find network centrality metrics to be good indicators of theory quality (measured by lint frequency):
The Spearman correlation of our six different centrality metrics was weaker than in similar experiments from software systems,
and with a coefficient of $s=\num[round-mode=places,round-precision=2]{\lintCorrelationSlocOptS}$,
the source lines of code metric exhibited a stronger correlation than all centrality metrics considered.
In contrast, network centrality metrics worked well in predicting the most important concepts within AFP entries:
Of the definitions deemed most important by entry authors,
\SI[round-mode=places,round-precision=1]{\bestPredOptRec}{\percent} could be identified
at a precision of \SI[round-mode=places,round-precision=1]{\bestPredOptPre}{\percent}
(optimal \fOne-score),
using in-degree centrality.
At the cost of a few percentage points of precision,
a second maximum of \SI[round-mode=places,round-precision=1]{\bestPredOptTwoRec}{\percent} recall can be achieved.

\keywords{Isabelle \and Archive of Formal Proofs \and Complex networks \and Dependency graph \and Formal entity network \and Centrality metrics \and Formalization quality.}
\end{abstract}
