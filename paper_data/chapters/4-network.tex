\section{Analysis of Formal Entity Networks}\label{sec:network}
Generally, we denote as a \emph{formal entity network} the graph of logically relevant entities in a formalization with their relationships.
Hence, nodes of the graph include types, constants and theorems.
Depending on the logic and system used,
other concepts may also be represented,
for example type-classes.
Edges model the relationships in between those entities,
and between entities and the syntactical structure (e.g., theory files).
Entity relationships exist between types, constants and types, theorems and constants, and theorems (proof dependencies).
For Isabelle,
we include locales and classes, locale dependencies,
and model sort constraints as well as class relationships.
Our syntactical structure consists of \emph{defining positions}
(i.e., the Isabelle source code from which entities are created),
theories, and sessions.
Depending on the research question, we consider sub-graphs of the full network, e.g., only entity nodes.

The full dependency graph of the Isabelle distribution and the AFP consists of \SI[round-mode=places,round-precision=1]{\numTotalNodes}{\million} entity nodes with \SI[round-mode=places,round-precision=1]{\numTotalEdges}{\million} edges between them.
When considering the \num{\numGraphs} nonempty sub-graphs for individual AFP entries (excluding the \texttt{slow} group, which contains a few particularly computationally extensive entries),
they make up for a total of \SI[round-mode=places,round-precision=1]{\numNodes}{\million} nodes and \SI[round-mode=places,round-precision=1]{\numEdges}{\million} edges.
In the following, we analyze this body of data to answer the following research questions:

\begin{enumerate}[label=\bfseries{RQ\arabic*}:, leftmargin=*]
\item Does the theorem dependency graph exhibit complex network properties?
\item How does formalization quality reflect on the theorem dependency network?
\item Can we detect important concepts within the dependency network automatically?
\end{enumerate}


\subsection{Dependency Graph as Complex Network}
To assess complex network properties of Isabelle and the AFP,
we analyze the full graph as well as the individual entry sub-graphs.
However, for the average path length, paths to elements of the object logic would largely determine the result.
Hence, we analyze the individual sub-graphs for small-world effects, and compare averages weighted by number of nodes.
In the undirected graph, the average shortest path length is \num[round-mode=places,round-precision=1]{\avgL},
which is quite close to the expected value of \num[round-mode=places,round-precision=1]{\avgLEr} for ER-graphs.
On average, the clustering coefficient is \num[round-mode=figures,round-precision=2]{\avgCC}
(i.e., about one third of a nodes' neighbors are also connected), which is orders of magnitudes larger than in ER-graphs (\num[round-mode=figures,round-precision=2]{\avgCCEr}) ---
confirming the small-world hypothesis.

\begin{knitrout}
\definecolor{shadecolor}{rgb}{0.969, 0.969, 0.969}\color{fgcolor}\begin{figure}[!htbp]

{\centering \includegraphics[width=\maxwidth]{figure/degree_full-1} 

}

\caption[In-degree distribution of full Isabelle/AFP dependency graph, on $\log$--$\log$ scale]{In-degree distribution of full Isabelle/AFP dependency graph, on $\log$--$\log$ scale. Scatter plot (left) with and CCDF (right), with fitted distributions.}\label{fig:degree_full}
\end{figure}

\end{knitrout}


In-degree frequencies of the full dependency graph is shown in \autoref{fig:degree_full} as a scatter plot (left).
The data roughly follows a linear line on the $\log$-$\log$ scale,
but there are some outliers.
The indicator line marks the fitted power-law distribution with a slope of $\alpha = \num[round-mode=places,round-precision=2]{\degreeFullAlpha}$.
On the right of \autoref{fig:degree_full},
the complementary cumulative distribution function (CCDF) of the empirical data and theoretical distributions (scaled to the empirical data at their minimal $k_\text{in}$) are shown.
We compare the scale-free distribution with common alternatives, namely the log-normal, exponential, and Poisson distribution.
The empirical data is close to the scale-free and log-normal distributions for smaller $k_\text{in}$,
but towards the tail it does not follow the theoretical values of any distribution closely.
This is not uncommon even for networks that are attributed scale-free structure:
Data-sets following a theoretical scale-free distribution very closely are found to be rare~\cite{ScalefreeRare2019Broido}.
In a likelihood ratio test between the two plausible distributions
(where \emph{likelihood} is the probability of the data given the distribution, and $\mathcal{R}$ the $\log$ of the ratio between the likelihoods),
the scale-free distribution is preferred with $\mathcal{R}=\num[round-mode=places,round-precision=1]{\degreeFullPlLnR}$.

\begin{knitrout}
\definecolor{shadecolor}{rgb}{0.969, 0.969, 0.969}\color{fgcolor}\begin{figure}[!htbp]

{\centering \includegraphics[width=\maxwidth]{figure/degree_ind_density-1} 

}

\caption[CCDF of entity in-degrees for $20$ randomly selected AFP entries (individual graphs), with fitted scale-free distribution ($\log$-$\log$ scale)]{CCDF of entity in-degrees for $20$ randomly selected AFP entries (individual graphs), with fitted scale-free distribution ($\log$-$\log$ scale).}\label{fig:degree_ind_density}
\end{figure}

\end{knitrout}

For dependency graphs of the individual entries
--- which are quite heterogeneous ---
a scale-free distribution is preferred only in \num{\degreeIndPlPrefNum} out of \num{\degreeIndNum} cases.
\autoref{fig:degree_ind_density} shows the complementary cumulative distributions of a random selection of $20$ entries (names truncated).
The larger entries tend to follow the distribution of the overall network,
whereas for the smaller entries, the tail of the distribution deviates.
This indicates that finite-size effects might skew the distribution,
which is often the case for scale-free networks~\cite{FiniteSize2021Serafino}.
Overall, the power-law parameter is similar, with a weighted average of $\alpha=\num[round-mode=places,round-precision=2]{\degreeIndAlpha}$ for the individual entries.

Hence, we expect that some results from complex network science carry over to our dependency graphs.
Thus, to answer our next research questions, we utilize the centrality metrics introduced in Section~\ref{sec:background} as they are helpful for complex networks in other settings.
We first analyze those metrics on their own:
\autoref{tab:metrics_correlation} shows how the centrality metrics are correlated with each other in the networks of individual entries,
as matrix of Spearman correlation coefficients $s$ (i.e., strength of the monotonic relation between two variables) ---
all are significant with $p<.001$, $n=\SI[round-mode=places,round-precision=1]{\metricCorrelationN}{\million}$.
Some correlations are expected, since they would also occur in random graphs.
In-degree is strongly correlated with closeness centrality
(any directly connected node has a distance of one)
as well as betweenness centrality
(an incoming edge means that a node is likely in a shortest path),
and moderately with eigenvector centrality
(incoming edges contribute to the rank).
Betweenness and closeness are also strongly related due to their similar nature.
Notably, there is also a fairly strong correlation between eigenvector and closeness centrality.
\DTLloaddb{metrics_correlation}{data/csv/gen/metrics_correlation.csv}

\begin{table}[!htbp]
    \centering
    \caption{Coefficients of Spearman correlation between centrality metrics for individual formal entity graphs, all significant at $p<.001$, $n=\SI[round-mode=places,round-precision=1]{\metricCorrelationN}{\million}$. Values above $s=0.5$ are printed bold.}
    \label{tab:metrics_correlation}
    \begin{tabular}{lrrrrrr}\toprule
    & $k_\textrm{in}$ & $k_\textrm{out}$ & $CC$ & $C_\lambda$ & $C_\textrm{dist}$  & $C_p$
    \DTLforeach*{metrics_correlation}{\out=out_degree,\in=in_degree,\eigenvector=eigenvector,\closeness=closeness,\betweenness=betweenness,\clustering=clustering,\rowname=rowname}{
        \DTLiffirstrow{\\\cmidrule{1-7}}{\\}
        \rowname & \in & \out & \clustering & \eigenvector & \closeness & \betweenness
    }
    \\\bottomrule
    \end{tabular}
\end{table}


\subsection{Assessing Formalization Quality}
Measuring formalization quality is a challenging task:
Manual analysis is quite labor-intensive and difficult,
as it is often not obvious on the surface why something was formalized in a certain way
(provability usually being a key factor).
Moreover, there are no defects (like in software systems) that would give an indication about quality.
Metrics such as change frequencies would also not provide meaningful insights due to the journal character of the AFP\@.
Therefore, we have to rely on static analysis,
and utilize the recently developed \emph{Isabelle linter}~\cite{Linter2022Megdiche}
which can identify problematic anti-patterns (\emph{lints}),
to measure formalization quality.
However, almost all available lints are concerned with proofs, so they only uncover specific kinds of problems.
Hence, lints do not indicate structural problems of specific entities,
but rather serve as a general quality indicator for whole theories or entries.
As quality metric, we use \emph{lint frequency}, i.e., the number of lints per \emph{source line of code} (SLOC).
We compare the lint frequency to our centrality metrics in a range of scenarios,
using the total, mean, and maximum as aggregation for the centrality scores on the formal entity network.
For a thorough analysis, we also employ these metrics on the graph induced by theories and defining positions.
Moreover, we utilize the fractal dimension (approximated by box covering~\cite{FractalDimension2007Song}),
which is found to be a particularly good quality indicator in software systems
--- Tosun et al.~\cite{NetworkDefects2009Tosun} use a linear regression in the $\log$-$\log$ space to compute the power-law coefficient,
which we include as $d_B^\text{lin}$ in addition to the proper stochastic fit~\cite{PowlawEmpiric2009Clauset}.
To evaluate how useful the centrality metrics are,
we compare their performance to that of the SLOC metric.
Since using the frequency already adjusts for size,
there is no inherent correlation between theory size and lint frequency,
but larger files could potentially be an indicator of lacking structure.

\DTLloaddb{centrality_metrics_cor}{data/csv/gen/centrality_metrics_cor.csv}

\begin{table}[!htbp]
    \centering
    \caption{Coefficients of Spearman correlation between centrality metrics and lint frequency in different scenarios, significant at the $p=0.01$ level. Values above $s=0.25$ are printed bold.}
    \label{tab:centrality_metrics_cor}
    \begin{tabular}{llcrrrrrr}
    \toprule
    Lint Frequency & Subgraph & Aggregation & $k_\textrm{in}$ & $k_\textrm{out}$ & $CC$ & $C_\lambda$ & $C_\textrm{dist}$  & $C_p$ \\
    \midrule
    \multirow{6}{*}{per entry} & \multirow{3}{*}{entities} \\[-\normalbaselineskip]
    \DTLforeach*[\DTLiseq{\granularity}{entry} \AND \DTLiseq{\nodes}{entity} \AND \DTLiseq{\score}{mean}]{centrality_metrics_cor}{\score=score,\nodes=nodes,\granularity=granularity,\out=out_degree,\in=in_degree,\eigenvector=eigenvector,\closeness=closeness,\betweenness=betweenness,\clustering=clustering}{ & & \score & \in & \out & \clustering & \eigenvector & \closeness & \betweenness \\}
    \DTLforeach*[\DTLiseq{\granularity}{entry} \AND \DTLiseq{\nodes}{entity} \AND \DTLiseq{\score}{total}]{centrality_metrics_cor}{\score=score,\nodes=nodes,\granularity=granularity,\out=out_degree,\in=in_degree,\eigenvector=eigenvector,\closeness=closeness,\betweenness=betweenness,\clustering=clustering}{ & & \score & \in & \out & \clustering & \eigenvector & \closeness & \betweenness \\}
    \DTLforeach*[\DTLiseq{\granularity}{entry} \AND \DTLiseq{\nodes}{entity} \AND \DTLiseq{\score}{max}]{centrality_metrics_cor}{\score=score,\nodes=nodes,\granularity=granularity,\out=out_degree,\in=in_degree,\eigenvector=eigenvector,\closeness=closeness,\betweenness=betweenness,\clustering=clustering}{ & & \score & \in & \out & \clustering & \eigenvector & \closeness & \betweenness \\\cmidrule{2-9}}
    & \multirowcell{3}[0pt][l]{defining\\ positions} \\[-\normalbaselineskip]
    \DTLforeach*[\DTLiseq{\granularity}{entry} \AND \DTLiseq{\nodes}{position} \AND \DTLiseq{\score}{mean}]{centrality_metrics_cor}{\score=score,\nodes=nodes,\granularity=granularity,\out=out_degree,\in=in_degree,\eigenvector=eigenvector,\closeness=closeness,\betweenness=betweenness,\clustering=clustering}{ & & \score & \in & \out & \clustering & \eigenvector & \closeness & \betweenness \\}
    \DTLforeach*[\DTLiseq{\granularity}{entry} \AND \DTLiseq{\nodes}{position} \AND \DTLiseq{\score}{total}]{centrality_metrics_cor}{\score=score,\nodes=nodes,\granularity=granularity,\out=out_degree,\in=in_degree,\eigenvector=eigenvector,\closeness=closeness,\betweenness=betweenness,\clustering=clustering}{ & & \score & \in & \out & \clustering & \eigenvector & \closeness & \betweenness \\}
    \DTLforeach*[\DTLiseq{\granularity}{entry} \AND \DTLiseq{\nodes}{position} \AND \DTLiseq{\score}{max}]{centrality_metrics_cor}{\score=score,\nodes=nodes,\granularity=granularity,\out=out_degree,\in=in_degree,\eigenvector=eigenvector,\closeness=closeness,\betweenness=betweenness,\clustering=clustering}{ & & \score & \in & \out & \clustering & \eigenvector & \closeness & \betweenness \\\cmidrule{1-9}}
    \multirow{7}{*}{per theory} & \multirow{3}{*}{entities} \\[-\normalbaselineskip]
    \DTLforeach*[\DTLiseq{\granularity}{theory} \AND \DTLiseq{\nodes}{entity} \AND \DTLiseq{\score}{mean}]{centrality_metrics_cor}{\score=score,\nodes=nodes,\granularity=granularity,\out=out_degree,\in=in_degree,\eigenvector=eigenvector,\closeness=closeness,\betweenness=betweenness,\clustering=clustering}{ & & \score & \in & \out & \clustering & \eigenvector & \closeness & \betweenness \\}
    \DTLforeach*[\DTLiseq{\granularity}{theory} \AND \DTLiseq{\nodes}{entity} \AND \DTLiseq{\score}{total}]{centrality_metrics_cor}{\score=score,\nodes=nodes,\granularity=granularity,\out=out_degree,\in=in_degree,\eigenvector=eigenvector,\closeness=closeness,\betweenness=betweenness,\clustering=clustering}{ & & \score & \in & \out & \clustering & \eigenvector & \closeness & \betweenness \\}
    \DTLforeach*[\DTLiseq{\granularity}{theory} \AND \DTLiseq{\nodes}{entity} \AND \DTLiseq{\score}{max}]{centrality_metrics_cor}{\score=score,\nodes=nodes,\granularity=granularity,\out=out_degree,\in=in_degree,\eigenvector=eigenvector,\closeness=closeness,\betweenness=betweenness,\clustering=clustering}{ & & \score & \in & \out & \clustering & \eigenvector & \closeness & \betweenness \\\cmidrule{2-9}}
    & \multirowcell{3}[0pt][l]{defining\\ positions} \\[-\normalbaselineskip]
    \DTLforeach*[\DTLiseq{\granularity}{theory} \AND \DTLiseq{\nodes}{position} \AND \DTLiseq{\score}{mean}]{centrality_metrics_cor}{\score=score,\nodes=nodes,\granularity=granularity,\out=out_degree,\in=in_degree,\eigenvector=eigenvector,\closeness=closeness,\betweenness=betweenness,\clustering=clustering}{ & & \score & \in & \out & \clustering & \eigenvector & \closeness & \betweenness \\}
    \DTLforeach*[\DTLiseq{\granularity}{theory} \AND \DTLiseq{\nodes}{position} \AND \DTLiseq{\score}{total}]{centrality_metrics_cor}{\score=score,\nodes=nodes,\granularity=granularity,\out=out_degree,\in=in_degree,\eigenvector=eigenvector,\closeness=closeness,\betweenness=betweenness,\clustering=clustering}{ & & \score & \in & \out & \clustering & \eigenvector & \closeness & \betweenness \\}
    \DTLforeach*[\DTLiseq{\granularity}{theory} \AND \DTLiseq{\nodes}{position} \AND \DTLiseq{\score}{max}]{centrality_metrics_cor}{\score=score,\nodes=nodes,\granularity=granularity,\out=out_degree,\in=in_degree,\eigenvector=eigenvector,\closeness=closeness,\betweenness=betweenness,\clustering=clustering}{ & & \score & \in & \out & \clustering & \eigenvector & \closeness & \betweenness \\\cmidrule{2-9}}
    & theories \\[-\normalbaselineskip]
    \DTLforeach*[\DTLiseq{\granularity}{theory} \AND \DTLiseq{\nodes}{theory}]{centrality_metrics_cor}{\score=score,\nodes=nodes,\granularity=granularity,\out=out_degree,\in=in_degree,\eigenvector=eigenvector,\closeness=closeness,\betweenness=betweenness,\clustering=clustering}{ & & \score & \in & \out & \clustering & \eigenvector & \closeness & \betweenness}\\
    \bottomrule
    \end{tabular}
\end{table}
\autoref{tab:centrality_metrics_cor} shows the Spearman correlation for different aggregations of the network metrics.
At our significance level of $p=0.01$,
most metrics are slightly correlated with lint frequency,
however the strength peaks at $s=\num[round-mode=places,round-precision=2]{\lintCorrelationOptS}$ for betweenness centrality in defining positions, using the total centrality per theory;
several other metrics are close contenders.
Results for the global metrics are shown in \autoref{tab:other_metrics_cor}.
The fractal dimension performs similar to the best centrality metrics,
with a correlation coefficient of $s=\num[round-mode=places,round-precision=2]{\lintCorrelationDlOptS}$ in the optimal case.
The difference between the computation methods of the fractal dimension is small,
though the regression calculation is slightly better.
In contrast, the simple SLOC metric performs significantly better:
The coefficient of $s=\num[round-mode=places,round-precision=2]{\lintCorrelationSlocOptS}$ (in the per-theory scenario) is stronger than any other correlation.
Hence, the centrality metrics considered here are not good quality indicators.
\DTLloaddb{other_metrics_cor}{data/csv/gen/other_metrics_cor.csv}

\begin{table}[!htbp]
    \centering
    \caption{Coefficients for Spearman correlation between global metrics and lint frequency in different scenarios. Values above $0.25$ are printed bold.}
    \label{tab:other_metrics_cor}
    \begin{tabular}{lcrrr}\toprule
    Score & Aggregation & SLOC & $d_B$ & $d_B^\text{lin}$\\
    \midrule
	\multirow{4}{*}{per entry} & \\[-\normalbaselineskip]
	\DTLforeach*[\DTLiseq{\score}{entry}]{other_metrics_cor}{\dl=d_l,\dllin=d_l_lin,\score=score,\sloc=sloc,\agg=aggregation}{\DTLiffirstrow{}{\\} & \agg & \sloc & \dl & \dllin}
    \\\cmidrule{1-5}
    \DTLforeach*[\DTLiseq{\score}{theory}]{other_metrics_cor}{\dl=d_l,\dllin=d_l_lin,\score=score,\sloc=sloc,\agg=aggregation}{per theory & \agg & \sloc & \dl & \dllin}
    \\\bottomrule
    \end{tabular}
\end{table}


\subsection{Predicting Important Concepts}
Knowing which concepts
(types, constants, etc.)
are important in a formalization is essential to understand the material quickly and effectively,
but so far, such overviews are not available.
For a small set of developments, this may be provided by hand.
However, to be feasible on a large scale, and for interactive tooling,
automatic extraction of important concepts is necessary.
Research in software systems suggests that centrality metrics of the dependency network could be a good indicator for identifying the most important concepts,
in particular closeness and in-degree centrality~\cite{DefectsMetrics2008Zimmermann}.

However, entities of a formal entity network do not directly relate to concepts.
The higher-level commands one uses to define a concept can introduce many entities,
e.g., the \texttt{datatype} command generates multiple types and constants to represent the potentially recursive datatype as a HOL type.
Hence, we consider as a concept all defining positions that introduce a type, constant, class, or locale;
from that, we select the entity that best matches the source code,
e.g., the type entity with the user-supplied name for a recursive datatype definition.

To collect ground truth data, we asked AFP authors
(who listed an e-mail address as affiliation)
for their opinion on the five most important concepts
and the five theorems that were most central to the reasoning in their formalization.
We included a list of unordered suggestions
(generated by the top ten entities each of several graph metrics)
to prompt responses,
but prominently stated that elements not in this list should be considered as well.
In total, we sent \num{\numEmails} e-mails (\num{\numEmailsUnreachable} addresses were unreachable) and received replies from \num{\numReplies} authors,
of which \num{\numAnswers} contained answers suitable for our analysis
(i.e., listing specific entities).

To evaluate our centrality metrics,
we compute the match of the top-$N$ elements with the ground truth
in terms of precision
(how many of the selected elements are relevant)
and recall
(how many of the relevant elements were selected).
The harmonic mean of precision and recall (called \fOne-score) is shown in \autoref{fig:metrics_f1} for a varying number of $N$,
in four different scenarios.
When predicting the entities representing important concepts,
the \fOne-score peaks for $N$ in the range of \numrange{5}{15} for all metrics.
In-degree and closeness centrality consistently achieve the highest score,
similar to what was found in software systems~\cite{DefectsMetrics2008Zimmermann}.
When only the defining positions of concepts are considered in the score calculations
(to mitigate the error that comes with selecting an entity to define a concept),
a slightly higher \fOne-score of at most \num[round-mode=places,round-precision=2]{\bestPredOptF}
is reached.
Otherwise, the results are the same.
\begin{knitrout}
\definecolor{shadecolor}{rgb}{0.969, 0.969, 0.969}\color{fgcolor}\begin{figure}[!htbp]

{\centering \includegraphics[width=\maxwidth]{figure/metrics_f1-1} 

}

\caption[$F_1$-scores for network centrality metrics in different scenarios by number of concepts selected, with marked maximum values]{$F_1$-scores for network centrality metrics in different scenarios by number of concepts selected, with marked maximum values.}\label{fig:metrics_f1}
\end{figure}

\end{knitrout}

In contrast, the prediction for central theorems scores much lower:
The \fOne-score is well below $0.1$ for most metrics.
Only the betweenness centrality performs significantly better and peaks at about \num[round-mode=places,round-precision=2]{\bestPredThmOptF},
both when predicting the full dependency graph, and when only the sub-graph of theorem entities is considered
(the only difference is that the peak is much sharper in the latter case).
The absolute value may be somewhat skewed as the definition of \enquote{central theorem} was not clear to all participants who contributed the ground truth data,
but the large relative difference shows that betweenness centrality is the key metric for central theorems.
An intuitive explanation is that central theorems are required in the transitive closure of proof dependencies of many theorems,
and thus appear in many shortest paths.


\begin{knitrout}
\definecolor{shadecolor}{rgb}{0.969, 0.969, 0.969}\color{fgcolor}\begin{figure}[!htbp]

{\centering \includegraphics[width=\maxwidth]{figure/relevant_concepts_prc-1} 

}

\caption[Precision-recall curve for prediction of defining positions for important concepts, with varying number of top-$N$ elements selected]{Precision-recall curve for prediction of defining positions for important concepts, with varying number of top-$N$ elements selected.}\label{fig:relevant_concepts_prc}
\end{figure}

\end{knitrout}

Since predicting the correct defining positions is our most important use-case,
we analyze the performance characteristic for that scenario in \autoref{fig:relevant_concepts_prc}.
Overall, 
the performance is strictly better where optimal \fOne-score is higher in nearly all cases,
i.e., in-degree is the best metric at every operation point.
The in-degree curve starts out at a precision of \SI[round-mode=places,round-precision=1]{\bestPredMaxPre}{\percent} (recall \SI[round-mode=places,round-precision=1]{\bestPredMaxPreRec}{\percent}), then sharply decreases for increasing $N$.
\fOne-score is optimal for $N=\bestPredOptN$, at \SI[round-mode=places,round-precision=1]{\bestPredOptPre}{\percent} precision and \SI[round-mode=places,round-precision=1]{\bestPredOptRec}{\percent} recall.
A second good operating point
(optimal when recall is considered more important as precision, e.g., by a factor of two in the $F_2$-score)
is at $N=\bestPredOptTwoN$,
where precision is only a little lower (\SI[round-mode=places,round-precision=1]{\bestPredOptTwoPre}{\percent}) but recall much higher with \SI[round-mode=places,round-precision=1]{\bestPredOptTwoRec}{\percent}.
However, the final choice of $N$ may depend on other user interface design considerations.
The relatively low precision is acceptable for a generated overview,
and due to the nature of our ground truth data (restricting to the top few elements),
many high-ranking concepts not in the ground truth set are likely to be relevant ---
several responses stated that it was difficult to limit themselves to such a small set.
